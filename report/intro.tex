\section{Introduction}
During the class of Software-Engineering we were introduced in the language of java as an object-oriented 
programming language. The main advantages are the expandibility and the very comprehensive looking of 
the code.

In order to reinforce our knowledge in object-oriented programming we were asked to design a software
which manages a food-delivery service. There are several stakeholders in such a system, like clients, 
restaurants, etc.. The software we created is a first approach to handle such a system which we
designed to respect the open-closed principle in order to be easily extensible to further functions.
Following our process of implementation is described, as well as the important choices we made in 
terms of software design. In the second the problems in terms of coding that we encountered are 
discussed. Following to this, the use of JUnit-tests and a \textit{testing-coverage} is described.
Furthermore there is user's guide added to this report to guarantee a smooth entry in the use of the
software.
