\section{Coding problems and their solutions}
\label{sec:coding_problems_and_their_solutions}

\subsection{Allocate a \textsc{Courier} to an order}
\label{sub:allocate_a_courier_to_an_order}

Beside the use of the Strategy-pattern to choose between the different delivery policy, there were
some other important choices for the implementation of this feature. The \textsc{allocate} method 
is written in the two different implementations of the Delivery interface. Since the system is 
just used locally by one user at a time, it was not possible to create a interaction of ordering
by a customer and the accepting of the delivery by a courier. In order to imitate some this 
behaviour we worked with a accepting-probability for the courier, which can be set in the core of
the \textsc{MyFoodoar} system. 

The fact that a courier can reject the delivery of an order caused a problem when allocating the 
courier to the order. We solved this problem with a \textit{while-loop} that only ends when
a courier is found who is available and accept the order. In order to avoid that one courier who
has rejected an order does not get the same order once again, he is added to a temporary list of
couriers. We find this problem while testing the code with the \textit{JUnit-Tests}. The tests 
also revealed an error in the allocating algorithme. Before we did not check if the choosen 
courier was \textit{On-Duty}.

\lstinputlisting[firstline=20, lastline=35]{../../src/system/FairDelivery.java} 
