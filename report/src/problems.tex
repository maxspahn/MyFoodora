\section{Coding problems and their solutions}
\label{sec:coding_problems_and_their_solutions}

\subsection{Allocate a \textsc{Courier} to an order}
\label{sub:allocate_a_courier_to_an_order}

Beside the use of the Strategy-pattern to choose between the different delivery policy, there were
some other important choices for the implementation of this feature. The \textsc{allocate} method 
is written in the two different implementations of the Delivery interface. Since the system is 
just used locally by one user at a time, it was not possible to create a interaction of ordering
by a customer and the accepting of the delivery by a courier. In order to imitate somehow this 
behavior we worked with an accepting-probability for the courier, which can be set in the core of
the \textsc{MyFoodora} system. 

The fact that a courier can reject the delivery of an order caused a problem when allocating the 
courier to the order. We solved this problem with a \textit{while-loop} that only ends when it founds
a courier who is available and accept the order. In order to avoid that one courier who
has rejected an order get the same order once again, he is added to a temporary list of
couriers. We found this problem while testing the code with the \textit{JUnit-Tests}. The tests 
also revealed an error in the allocating algorithm. Before we did not check if the chosen 
courier was \textit{On-Duty}.

\lstinputlisting[firstline=20, lastline=35]{./../src/system/FairDelivery.java} 

\subsection{Shipped order sorting policy}
\label{sub:shipped_order_sorting_policy}

One requirement of the system was to sort the shipped meals and single-items according the 
number sold. Our first approach to simply store the number as an attribute to each item
was fast proven being limited, because the program was not able to make the distinction
between which restaurant had sold the concerned item how many times. We wanted to 
make the sorting algorithm accessible to every restaurant manager so that he can see  the 
sorted list of the items sold in his restaurant. So we created a more sophisticated 
structure, that includes the abstract class \lstinline|SortPolicy| with two attributes,
one is the \textit{item-name} and the other is a \textit{count}, an integer storing the number
of sold items.
This class implements the interface \lstinline|Comparable <SortPolicy>| and comes with 
an implementation of the method \lstinline|compareTo(SortPolicy sort)| to compare two
objects according to their count.

\lstinputlisting[firstline=27, lastline=32]{./../src/system/SortPolicy.java}

There are three subclasses of \lstinline|SortPolicy|: \lstinline|SingleItemSort|, \lstinline|HalfMealSort| and \lstinline|FullMealSort|. Within each 
\lstinline|restaurant| there are three \lstinline|TreeSet<SortPolicy>|, one for each 
item-type. The advantage of this structure is that it is sorted immediately when adding a new 
item. When an order is completed all its items are added to the corresponding TreeSet.
If the item is already in the TreeSet its counter is increased by the number in the 
order. It was difficult to fully understand how this data-type works but it allowed us
to fulfill this requirement.

\subsection{Calculation of the earnings between two dates}
\label{calculation-between-2-dates}
Another requirement was to compute different earnings (like total income, profit...) between two dates chosen by the user. To do so, the idea was to sort all the orders according to the date they were made. Then, the difficulty was to find the orders which had been made between the two dates. We implemented this algorithm:
\lstinputlisting[firstline=1,lastline=34]{./Code/src/Code/algorithmFindProfit.txt}
The code corresponding to this algorithm can be found in the method \textit{getIncomeForPeriod(...)} of \textit{MyFoodora}.

\subsection{Exceptions}
\label{sub:exceptions}

In our first approach we tried to avoid any kind of \lstinline|Exception| because we were
not convinced of its utility. We handled the possibility that there might occur wrong inputs at
several places with simple \lstinline|System.out.println()|. At first this seemed to be a
sufficient solution. As we continued to implement further parts of the system we realized that it
was becoming more and more confusing where the \lstinline|System.out.println()| is actually 
invoked. The solution was obviously the implementation of \lstinline|Exception| and we sacrificed
some time in their creation. Later in the project-work we realized that this choice was the right
one since the implementation of the \textsc{Command-Line-User-Interface} and the \textsc{GUI} were
much easier this way. In total there exist 13 different \lstinline|Exceptions|.

\subsection{Serialization}
\label{sub:serialization}
After having finished most of our program, we were a bit frustrated because we felt like our code was not a ``real software''. Indeed, after the execution of the CLUI and the closure of our program, all the data created by the user vanished, which means that at the restart of the CLUI, everything created before did not exist anymore. To avoid this problem and to make our software more realistic, we decided to serialize all the attributes of MyFoodora at the shut down of the program (done by the command \textit{"q"}) in a file \textit{MyFoodoraDatas.txt}.
Hence at the launch of the software, 2	cases are taken into account in the constructor of MyFoodora:
\begin{itemize}
	\item{} This is the first launch and therefore there is no file containing serialized data, and the constructor initializes all the attributes of MyFoodora with default values.
	\item{} The file \textit{MyFoodoraDatas.txt} has already been created and the constructor initializes the attributes of MyFoodora by deserializing it. So all the attributes of MyFoodora are the same that there were before the last shut down. 	
\end{itemize}
\lstinputlisting[firstline=1,lastline=15]{./Code/src/Code/constructorMyFoodora.txt}
However, we had trouble to implement this serialization. At the beginning we thought that only the class \textit{MyFoodora} and those which are instantiated as attributes of it needed to implement the interface \textit{Serializable}. Nevertheless, it took time to understand that all the classes are ``connected'' with each other, and that all of them need to implement \textit{Serializable}.

\subsection{Use of GIT}
\label{sub:use_of_git}

Working in a group can quickly be problematic when working with code because several persons
may change the same part of code to realize different features of the system. In order to 
avoid this type of problem we decided to use git as a program to share the program. 
Furthermore it allowed us to keep track of the history of our changing and even note what
kind of problems we encountered when working with the program. Several times \textsc{GIT}
helped us to solve conflicts between different versions of the implementation.
\centerline{\url{https://github.com/maxspahn/MyFoodora}}
