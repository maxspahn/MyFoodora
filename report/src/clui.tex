\section{Command-line-user-interface}
\label{sec:command_line_user_interface}

A useful program needs an user interaction and the first approach is usually a command line 
interface. The use of this feature is explained in this part of the report.

\subsection{Structure and Handling arguments}
\label{sub:structure_and_handling_arguments}

The different required commands were easy to define since the whole system that had been build 
previously was about to be used in the interface. In the package \textsc{commandLineTool} we 
stored all functions in \textsc{Launch}. We decided not to use a pattern separating
visual functions from back-end functions, because we are convinced that it was not necessary for a
project of such a small size and it would not justify the additional work time that it might 
cost.

One major problem we encountered in the process was the treatment of the input arguments because 
the number varies depending on the command. When the user does type a command he creates a 
\textsc{String} that is temporarily stored by the program. The next step is the separation of 
the different arguments in a array of Strings. It was required to use quotation marks for the 
arguments in order to avoid that arguments with white spaces, such as restaurant names,  are cut
into multiple arguments.
We considered this processing not convincing because the user is consequently forced to put 
quotation marks around every argument, although it is only useful in very few cases. Hence, 
our implementation does avoid the use of arguments with white spaces, by for example working 
exclusively with the Usenet. This choice is arguably a bit ``dirty'' but allows to better 
imitate the traditional command line syntax known from \textsc{UNIX} and improves the efficiency 
while using the tool.

The input of dates and addresses was also an issue because the format is to be chosen. If the 
following format is not respected, the command does not work like it should.
\begin{enumerate}
	\item \textbf{Address: } x-coordinate,y-coordinate
	\item \textbf{Dates: } dd/mm/yyyy
\end{enumerate}

\subsection{Use and Testing}
\label{sub:use_and_testing}

To launch the tool you have to open the file \textsc{src/commandLineTool/Launch.java} and execute
it. The main method starts the interaction loop, in which the user can type the commands. To get
a list of all available commands one has to type \textsc{help}. 

Since it is very time intensive to test all the tool, we created test-files that
contain a list series of commands. To execute those commands one has to run the command 
\textit{runTest <fileName>}. By adding the fileName as an argument, it is possible to create 
several test-Scenarios without changing the program. Hence, the user can load multiple 
preexisting restaurants with the test-file \textit{my\textunderscore foodora.ini}.
The main scenario is stored in \textit{testScenario1.txt}. If you run the command
\textit{runTest} without any argument this test-case is automatically used as the default test
file.

\paragraph{Explanation of the two test-files}
\label{par:explanation_of_the_two_test_files}

\textsc{my\textunderscore foordora.ini}
\begin{itemize}[itemsep=0mm]
	\item login as manager
	\item register two managers
	\item logout
	\item register five restaurants, two couriers, seven customers
	\item login as restaurant ``chezmarie''
	\item add dish to the menu of ``chezmarie''
	\item logout
\end{itemize}

\textsc{testScenario1.txt}
\begin{itemize}[itemsep=0mm]
	\item login as manager
	\item add restaurant with the username le\_vieux\_campeur
	\item add a new courier laura mendez
	\item add a new costomer max spahn
	\item show the list of couriers
	\item show the list of restaurants
	\item show memu of a specific restaurant, fiveFields
	\item show the list of customers 
	\item logout as manager
	\item login as the new restaurant vieux
	\item add a dish to the menu of the restaurant
	\item show the updated menu of the restaurant
	\item create a full meal with the created items
	\item set this meal as local special
	\item logout as restaurant
	\item show the updated menu of the restaurant
	\item login as new customer max spahn
	\item create order and add items to the order
	\item finish the order
	\item logout as customer
	\item login as manager
	\item show the total profit of the system 
	\item show total profit between 1 March and 1 May
	\item set the profit policy to deliverCost
	\item set fees according to this police(target is the same profit as last month) 
	\item show the new fees
	\item set targets and show targets
	\item set fees according to the policy and show the new fees
        \item show the couriers sorted by the number of deliverd orders
        \item logout as manage
\end{itemize}

