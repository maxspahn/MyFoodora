\section{Command-line-user-interface}
\label{sec:command_line_user_interface}

A useful program needs an user interaction and the first approach is usually a command line 
interface. The use of this feature is explained in this part of the report.

\subsection{Structure and Handling arguments}
\label{sub:structure_and_handling_arguments}

The different required commands were easy to define since the whole system that had been build 
previously was about to be used in the interface. In the package \textsc{commandLineTool} we 
stored all functions in \lstinline|Launch|. We decided not to use a pattern separating
visual functions from back-end functions, because we are convinced that it was not necessary for a
project of such a small size and it would not justify the additional work time that it might 
cost.

One major problem we encountered in the process was the treatment of the input arguments because 
the number varies depending on the command. When the user does type a command he creates a 
\lstinline|String| that is temporarily stored by the program. The next step is the separation of 
the different arguments in a array of Strings. It was required to use quotation marks for the 
arguments in order to avoid that arguments with white spaces, such as restaurant names,  are cut
into multiple arguments.
We considered this processing not convincing because the user is consequently forced to put 
quotation marks around every argument, although it is only useful in very few cases. Hence, 
our implementation does avoid the use of arguments with white spaces, by for example working 
exclusively with the Usenet. This choice is arguably a bit ``dirty'' but allows to better 
imitate the traditional command line syntax known from \textsc{UNIX} and improves the efficiency 
while using the tool.

It was also necessary to deal with permission rights in the user interface because there are 
different type of users who can executed different actions, i.e. the restaurant cannot see
the number of delivered orders by one specific courier. To fullfill this we created methods for
every user type that check if the current user is of that type. If it is not the case, the 
permission to the command is denied and the user is informed about it. Furthermore the 
command help does not show commands that are not available for the current user.

The input of dates and addresses was also an issue because the format is to be chosen. If the 
following format is not respected, the command does not work like it should.
\begin{enumerate}
	\item \textbf{Address: } x-coordinate,y-coordinate
	\item \textbf{Dates: } dd/mm/yyyy
\end{enumerate}

\subsection{Use and Testing}
\label{sub:use_and_testing}

To launch the tool you have to open the file \textit{src/commandLineTool/Launch.java} and execute
it. The main method starts the interaction loop, in which the user can type the commands. To get
a list of all available commands one has to type \lstinline|help|. 

Since it is very time intensive to test all the tool, we created test-files that
contain a list series of commands. To execute those commands one has to run the command 
\textit{runTest <fileName>}. By adding the fileName as an argument, it is possible to create 
several test-Scenarios without changing the program. Hence, the user can load multiple 
preexisting restaurants with the test-file \textit{my\textunderscore foodora.ini}.
Otherwise, it is possible to navigate the software by directly login, using the following datas:
\begin{itemize}
	\item{}Username: \textit{sparrowj}, Password: \textit{blackPearl} to login as a manager (the CEO).
	\item{}Username: \textit{batona}, Password: \textit{wer123} to login as a customer.
	\item{}Username: \textit{fiveFields}, Password: \textit{5fs} to login as a restaurant.
	\item{}Username: \textit{valmontb}, Password: \textit{1234} to login as a courier.
\end{itemize}

\subsection{Test Scenario}
\label{sub:test_scenario}

\textbf{It is crucial to this scenario that it is executed in order because some of the created items and 
users are used in the following commands.} Furthermore whenever you shut down the program do it by
typing \lstinline|q| in order to serialize the data. This can be done after each step, then the 
serie of commands can be continued.

\begin{enumerate}[itemsep=0mm]
	\item start the program and type \lstinline|help| to see the available options.
		Note that only those commands are shown for which the current user has the 
		permission, i.e. you are not allowed to see the profits before the login as 
		manager.
	\item load several users by typing \lstinline|runtest my_foodora.ini|. This command loads 
		some users to system, you can see the serie of commands printed in the console.
	\item shut down the system by typing \lstinline|q|. This stops the main-method.
	\item restart the program and execute \lstinline|runtest my_foodora.ini|. The system's 
		status has been saved thanks to the \textsc{Serialization} \ref{sub:serialization}.
		So when the program tries
		to create the users, it fails because they already exist. The error messages are 
		printed to the console.
	\item show the users by typing \lstinline|runtest showScenario.txt|. Three additional users
		are added. Then the list of couriers is printed to the console. Then that of the 
		restaurants, followed by the Menu of the restaurant \textsc{fiveFields} and a list of
		the customers.
	\item add some items to the menu of a restaurant that was created previously by running the 
		command \lstinline|runtest addItemsScenario.txt|. After the creation of three new 
		singleItems the updated menu is printed to the console. Afterwards a meal is created
		containing the new items. It is set to be the offer of the week for this restaurant.
		At the end you see once again the menu of the restaurant.
	\item create an order and finish it by typing \lstinline|runTest orderScenario.txt|. It 
		creates a new order with the name \textit{testOrder} and adds four different items, 
		including singleItems and meals. The order is shown and ``paid''. Then the history
		of complete orders for this customer is printed to the console before the logout.
	\item test if the wrong user could modify or see the order of another user by typing
		\lstinline|runtest order2Scenario.txt|. User \textsc{batona} creates an order but 
		does not finish it. Then the user \textsc{maxspahn3} tries to print it but got his
		access is denied, the same is valid for the modification of the order. Then the 
		user \textsc{batona} logs in again to finish his order.
	\item test the delivery system by typing \lstinline|runtest findDelScenario.txt|. It creates
		an order in the restaurant \textsc{typingRoom}. Then you login as manager to 
		associate a courier to this order. The asociated courier is printed to the console 
		(he is seen to be \textit{off-duty} because he is just delivering the order). 
		The courier is then set to be \textit{on-duty} once again. The whole process is 
		usually done by the system, where the order would now be added to the history of 
		orders. Since it is not the case, the order is removed from the system afterwards.
	\item check the profits by typing \lstinline|runtest profitScenario.txt|. First the customer
		batona creates an order and sets the completion date to \textsc{03/02/2015}. You can 
		see in the history of orders that it has been added. Then you login as a manager to 
		see different profits of the system, the time period is printed to the console.
	\item set the fees according to the different targetpolicies by typing 
		\lstinline|runtest targetScenario.txt|. It first sets the target policy to
		\textit{deliveryCost-based} and computes the fees according to this policy. As 
		target the profit of the previous month is used. Then the target is set manually and
		the fees are once again computed and printed to the console. Afterwards the 
		target policy is changed ot \textit{serviceFee-based}. When computing the fees, there
		occures an error, because the target cannot be fullfilled with the given values for 
		the two other values (markup-Percentage and delivery-Cost).
	\item notify customers about special offers in restaurants by typing
		\lstinline|runtest notifyScenario.txt|. First you see the notifications about the 
		special offer set previously in \textit{addItemsScenario.txt}. Then the restaurant
		\textsc{typingRoom} sets another specialoffer \textsc{fiveFields} changes its
		default discount. The user \textsc{batona} will recive two notifications. When he 
		calls \lstinline|getNotifications| a second time, they have disapperead to 
		simulate thate they were read.
			
	
		
\end{enumerate}

