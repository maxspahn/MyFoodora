\section{Conclusion}
In conclusion, this course was an real opportunity to make a complete software development and to learn a lot of notions in JAVA and Oriented Object Programming.We learnt to effectively design in an OPEN-CLOSE way thanks to the implementation of all the design patterns. It would be crazy to try to make such a big project without using specific design patterns to simplify the code structure. Moreover, we trully understood how tests are important, and that programming cannot be done without them. We spent a lot of time writinq the JUnit-Tests, but it was worth it.

As the project really motivated us, we wanted to go further than the requirements in order to make our software more convenient and realistic. That is why we spent time thinking about the SERIALIZATION, some algorithms, the scenarios and the GUI.

Nevertheless, we could have gone even further with more time. For example, we could have use threads so as to simulate the use of our software by several users at the same time. We also thought to create a website which can be accessed by several users at the same time (which is the last step to have a finished software), but it would require another course to do it.


To finish, thanks to Mr Ballarini for giving this course and helping us during the tutorials, and to Mr Lapitre for explaining the project and correcting it.