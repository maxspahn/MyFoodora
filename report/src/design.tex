\section{Design and the \textit{open-closed-principle}}
One main purpose of the use of java and object-oriented programming is that the code is easily
extended so new features and functions can easily be added to an already working system.
Nevertheless, one has to respect some rules of programming in order to benefit from this
advantage, i.e. the use of design patterns.
In our \textit{MyFoodora} we used several of these patters which are explained in this section.

\subsection{Package Management}
\label{sub:package_management}

Having read the project requirements for the first time we figured out that there were several
different parts which can be treated separately. This logical separation of different parts of the
system was used to organize our packages as well as the division of the work for the two members
of the group. The following main packages exist in the project:
\begin{itemize}
	\item{\textbf{system}} Contains the main functionality of \textit{MyFoodora}.
	\item{\textbf{user-management}} Manages users of the systems and implements functions
		to user-types.
	\item{\textbf{restaurant}} Manages the creation of meals and single-Items and do the
		pricing.
	\item{\textbf{commandLineTool}} Adds a command line user interface to the system.
	\item{\textbf{GUI}} Adds a graphical user interface to the system.
	\item{\textbf{several Test packages}} According their name they provide test for the
		previous packages.
\end{itemize}

The following diagram shows the interaction of the first three packages (The interfaces and
testings do obviously interact with all the other ones).

\begin{figure}[H]
	\centering
	\includegraphics[width=1\linewidth]{./ima/packages.jpg}
	\caption{Package-interactions, class \textit{MyFoodora} excluded to simplify}
	\label{fig:packages}
\end{figure}

Graphic \ref{fig:packages} shows the three main packages that form the core of our system. All
interaction between the packages are displayed except from those with the class
\lstinline|MyFoodora|. The separation chosen allowed us to separately work on the
\textsc{user-management} and \textsc{restaurant} without depending on parts of the code of the
other package, because there is only one interaction between the two. In a third step we
implemented \textsc{system}, because it was the most sophisticated one and requires large parts of
the two other packages. \lstinline|Order| is the most connected class besides
\lstinline|MyFoodora|.
One might consider to place it in a of the other packages. We decided to store it in the system
because we wanted functions like the computation of an order's price to be part of the system. In
this manner we were able to put all the core functions together. 

\subsection{DesignPattern : Factory-Pattern}
\label{sub:designpattern_factory_pattern}

A major pattern to respect the \textit{open-closed-principle} is the Factory-Pattern because it
allows the code to be easily extended without modifying the existing code. It is used twice in
the system:
\begin{enumerate}
	\item Creation of Items for the Restaurant
	\item Creation of Users in the user-management
\end{enumerate}

\paragraph{Creation of Items for the Restaurant}

In order to handle the creation of \textsc{singleItems} and \textsc{meals} in the same method we 
decided to use a \textbf{Factory-Pattern} with an \textit{abstract factory} and a
\textit{factoryProducer}. 

\begin{figure}[H]
	\centering
	\includegraphics[width=1\linewidth]{./ima/restaurant_factory_pattern.jpg}
	\caption{Factory-Pattern in package \textsc{restaurant}}
	\label{fig:restaurant-factory-pattern}
\end{figure}

The \lstinline|Menu| is the center class of \textsc{restaurant} which has an instance of 
\lstinline|ItemFactoryProducer| to produce new items and store them in an Array-List. Following you
see the code to add a new item:

\lstinputlisting[firstline=226, lastline=249]{../../src/restaurant/Menu.java}

This method only takes two Strings as arguments, where one is the item-type and the other its
name. Depending on the type, different \textit{if-clauses} are triggered to create the new item.
This design allows the user to add another item-type, i.e. drinks, with no need to modify the
existing code. Another advantage of this design is that the programmer do not even has to add a new
factory to the core class \textsc{Menu} since they are stored in \textsc{ItemFactoryProducer}. In
this manner modification that might cause errors to the already existing system are avoided and
the code is only extended by adding \textit{else-if-clauses} and the new classes. 

Another decision that was made concerning the restaurant design can be observed in the
class-diagram \ref{fig:restaurant-factory-pattern}. In order to guarantee that a meal does not
contain two single-Items of the same type(Starter, Main-Dish, Dessert) we decided to give a meal an
attribute for each type instead of an \textit{ArrayList} with all single-Items. Because of that
choice we are not obliged to search in the list if one type already exists when adding a new item.
It seemed more practical and it is unlikely that another single-Item is added.

\paragraph{Creation of new \textit{Users}}
So as to handle the creation of the 4 different users, a \textbf{Factory method pattern} has been implemented.
\begin{figure}[H]
	\centering
	\includegraphics[width=1\linewidth]{./ima/userFactory.jpg}
	\caption{Factory method pattern for the creation of users}
	\label{fig:user-factory-pattern}
\end{figure}
As illustrated in the previous figure \ref{fig:user-factory-pattern}, it is composed of 2 abstract classes: the factory \lstinline|UserFactory| which contains the method \lstinline|createAccount()| and the class \lstinline|User|. Each subclass of \lstinline|UserFactory| has its own definition of the method \lstinline|createAccount()| so that \lstinline|CourierFactory| creates a new courier, \lstinline|ManagerFactory| a new manager... This factory method pattern is totally OPEN-CLOSE.
\subsection{DesignPattern : Strategy-Pattern}
\label{sub:designpattern_strategy_pattern}

\textsc{MyFoodora} does have several options where the user (in many cases a manager) can change
different behaviors of the program concerning choosing the \textsc{Courier}. In these
cases we decided to apply the Strategy-Pattern to respect the the \textit{open-closed-principle}.
There are three parts of the program in which we used the this Design-Pattern:
\begin{itemize}
	\item Delivery Policy, allocate a \textsc{Courier} to an order
	\item Fidelity Card, what kind of discount is applied with regard to the
		\textsc{Customer}
	\item Target Policy, compute the fees for the system 
\end{itemize}

\paragraph{Delivery Policy}
\label{par:delivery_policy}
A new delivery policy could be easily added, by creating a new implementation
\textsc{DeliveryPolicy}.

\begin{figure}[H]
	\centering
	\includegraphics[width=1\linewidth]{./ima/deliverypolicy.jpg}
	\caption{Delivery Policy}
	\label{fig:deliveryPolicy}
\end{figure}

\paragraph{Fidelity Card}
\label{par:fideltycard}

There a several different fidelity cards and one can easily imagine that further cards might be 
added in the future. Hence, we decided to use another Strategy-Pattern to implement the price of
of an order. 

\begin{figure}[H]
	\centering
	\includegraphics[width=1\linewidth]{./ima/fidelitycard.jpg}
	\caption{Fidelity Cards}
	\label{fig:fidelitycards}
\end{figure}

\subsection{Design-Pattern : Observer-Pattern}
\label{sub:designpattern_observer_pattern}
One of the requirements of the design brief was to enable the customers to receive notifications when a restaurant changes its meal of the week or its special discount. To do so, an \textbf{Observer pattern} has been implemented.

\begin{figure}[H]
	\centering
	\includegraphics[width=1\linewidth]{./ima/observerV2.jpg}
	\caption{Observer pattern for the notifications}
	\label{fig:./ima/observer}
\end{figure}

The class \textit{Restaurant} is the \textbf{concrete observable}, and the class \textit{Customer} is the \textbf{concrete observer}. One can notice that the concrete observable does not have a list containing all the observers, nor the methods to add or remove an observer. The observers are those who agreed to receive special offers, so we could have created a list containing all of them. Nevertheless, there is already a list containing all the customers as attribute of \textit{MyFoodora}, so rather than creating twice a list of customers, we chose to directly use the existing list of MyFoodora as list of observers. This is in the method \textit{notifyObservers()} that the spam-agreement is checked and the notification is sent only to those who have \textit{spamAgreement = true}.

\lstinputlisting[firstline=105,lastline=115]{./../src/user_management/Restaurant.java}
